% --- CHƯƠNG 2: QUY TRÌNH ---
\chapter{Quy trình review code}

Quy trình review code của nhóm được thực hiện theo 3 bước chính: Chuẩn bị, Thực hiện và Tổng hợp.

\section{Chuẩn bị}
\begin{itemize}
    \item \textbf{Nghiên cứu Checklist:} Nghiên cứu và hiểu rõ 66 mục trong sheet \texttt{"Check list"} trong file \texttt{"Code Review Report Sample.xlsx"} được cung cấp. Các mục này được chia thành 6 loại (DEVIATION, OMISSION, DEFECT, AMBIGUITY, REDUNDANCE, và các mục con).
    \item \textbf{Chuẩn bị công cụ:} Sử dụng Microsoft Excel để ghi lại các phát hiện (file \texttt{nhom46\_code\_review.xlsx}) và Visual Studio Code để xem xét mã nguồn Java.
    \item \textbf{Mục tiêu:} Đối tượng review là 22 file mã nguồn \texttt{.java} của một dự án Web E-commerce.
\end{itemize}

\section{Thực hiện}
\begin{itemize}
    \item Tiến hành đọc và phân tích thủ công từng file \texttt{.java}.
    \item Đối với mỗi file, duyệt qua 66 mục của checklist để xác định các vi phạm.
    \item Khi phát hiện vi phạm, ghi lại thông tin vào sheet Excel tương ứng với file đó, bao gồm:
    \begin{itemize}
        \item \textbf{Check code:} Mã lỗi vi phạm (ví dụ: 15).
        \item \textbf{Check code Description:} Mô tả của mã lỗi.
        \item \textbf{Line:} Dòng code xảy ra vi phạm.
        \item \textbf{Comment:} Mô tả chi tiết về lỗi vi phạm.
        \item \textbf{Suggestion / Fix ?:} Đề xuất cách sửa lỗi.
    \end{itemize}
\end{itemize}

\section{Tổng hợp}
\begin{itemize}
    \item Sau khi hoàn tất review 22 file, toàn bộ dữ liệu từ các sheet Excel được tổng hợp lại.
    \item Sử dụng thư viện \texttt{pandas} và \texttt{matplotlib} của Python để phân tích và trực quan hóa dữ liệu tổng hợp.
    \item Các biểu đồ thống kê được tạo ra để cung cấp cái nhìn tổng quan về chất lượng mã nguồn (xem Chương 4).
    \item Viết báo cáo để trình bày lại toàn bộ quy trình, các phát hiện nghiêm trọng và kết quả thống kê.
\end{itemize}

\newpage