% --- CHƯƠNG 5: KẾT LUẬN ---
\chapter{Kết luận}

Quá trình review code thủ công \textbf{22 file} mã nguồn Java đã phát hiện tổng cộng \textbf{304 lỗi} vi phạm checklist.
Số loại lỗi (theo Check code) riêng biệt: \textbf{33 lỗi}.
Phân tích thống kê cho thấy các vấn đề nghiêm trọng nhất không chỉ tồn tại mà còn lặp lại một cách có hệ thống, tập trung vào các nhóm lỗi:
\begin{enumerate}
    \item \textbf{DEFECT (65.7\%):} Chiếm đa số, bao gồm các lỗi nghiêm trọng nhất như SQL Injection, mật khẩu hardcode, và mật mã yếu.
    \item \textbf{DEVIATION (22.9\%):} Chủ yếu là vi phạm quy ước đặt tên (naming convention).
\end{enumerate}

Các file DAO (\texttt{BookDAO.java}, \texttt{UserDAO.java}, \texttt{SalesDAO.java}) và các Servlet (\texttt{BookUpdate.java}) là những nơi tập trung nhiều lỗi nhất và cũng là những nơi cần được ưu tiên sửa chữa.

\textbf{Đề xuất:} Dự án ở trạng thái \textbf{không an toàn} để triển khai (deploy) lên môi trường production.
Nhóm phát triển cần:
\begin{itemize}
    \item \textbf{Ưu tiên 1 (Critical):} Sửa toàn bộ các lỗ hổng SQL Injection bằng \texttt{PreparedStatement} và gỡ bỏ toàn bộ thông tin nhạy cảm (credentials) ra khỏi mã nguồn.
    \item \textbf{Ưu tiên 2 (High):} Nâng cấp dịch vụ mã hóa mật khẩu.
    \item \textbf{Ưu tiên 3 (Medium/Low):} Thực hiện refactor để xử lý NullPointerException, đóng tài nguyên (resource leaks) và chuẩn hóa lại quy ước đặt tên.
\end{itemize}