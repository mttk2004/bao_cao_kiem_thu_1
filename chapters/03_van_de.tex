% --- CHƯƠNG 3: VẤN ĐỀ CHÍNH ---
\chapter{Các vấn đề và đề xuất}
Qua quá trình review, đã phát hiện tổng cộng \textbf{166 lỗi} trên 22 file. Dưới đây là các vấn đề nổi bật và nghiêm trọng nhất được phân loại theo mức độ ảnh hưởng.

\section{Lỗi Bảo mật Nghiêm trọng: SQL Injection (Hạng: Critical)}
\textbf{Vấn đề:} Đây là lỗi nghiêm trọng nhất, xuất hiện lặp lại ở hầu hết các class DAO (Data Access Object) như \texttt{BookDAO.java}, \texttt{UserDAO.java}, \texttt{OrderDAO.java}, và \texttt{SalesDAO.java}. Mã nguồn đang sử dụng phương thức cộng chuỗi (string concatenation) để xây dựng các câu lệnh SQL, cho phép kẻ tấn công chèn mã SQL độc hại và truy cập trái phép vào cơ sở dữ liệu.

\textbf{Ví dụ (từ \texttt{UserDAO.java}):}
\begin{lstlisting}[language=Java, caption={Lỗ hổng SQL Injection trong UserDAO.java (Line 25)}, firstnumber=25]
// Lỗi: Cộng chuỗi trực tiếp từ input của người dùng
String query = "SELECT * FROM user WHERE email = '" + email + "'";
// ...
ResultSet rs = st.executeQuery(query);
\end{lstlisting}

\textbf{Đề xuất:} Ngay lập tức thay thế \textbf{tất cả} các câu lệnh SQL cộng chuỗi bằng cách sử dụng \texttt{PreparedStatement}. \texttt{PreparedStatement} sẽ tự động xử lý và vô hiệu hóa các ký tự đặc biệt, ngăn chặn hoàn toàn các cuộc tấn công SQL Injection.

\textbf{Ví dụ sửa lỗi:}
\begin{lstlisting}[language=Java, caption={Sửa lỗi bằng PreparedStatement}]
// Sửa lỗi: Sử dụng PreparedStatement với tham số '?'
String query = "SELECT * FROM user WHERE email = ?";
PreparedStatement ps = connection.prepareStatement(query);
ps.setString(1, email);
ResultSet rs = ps.executeQuery();
\end{lstlisting}

\section{Lỗi Bảo mật Nghiêm trọng: Lộ thông tin nhạy cảm (Hạng: Critical)}
\textbf{Vấn đề:} File \texttt{ModelManager.java} (class quản lý kết nối CSDL) chứa thông tin đăng nhập (username và password) được \textbf{hardcode} (mã hóa cứng) trực tiếp trong mã nguồn. Điều này cực kỳ nguy hiểm vì bất kỳ ai có quyền truy cập vào mã nguồn đều có thể thấy và đánh cắp thông tin đăng nhập CSDL.

\textbf{Ví dụ (từ \texttt{ModelManager.java}):}
\begin{lstlisting}[language=Java, caption={Thông tin nhạy cảm bị hardcode (Line 8-10)}, firstnumber=8]
private static final String url = "jdbc:mysql://localhost:3306/bookstore";
private static final String dbuser = "root";
// Lỗi: Mật khẩu bị hardcode (dù đã bị comment)
// private static final String dbpass = "123456";
private static final String dbpass = "admin";
\end{lstlisting}

\textbf{Đề xuất:} Xóa ngay lập tức thông tin đăng nhập khỏi mã nguồn. Sử dụng các cơ chế an toàn hơn như biến môi trường (environment variables) hoặc file cấu hình (ví dụ: \texttt{.properties}, \texttt{.env}) để lưu trữ và tải thông tin này khi ứng dụng khởi chạy.

\section{Lỗi Bảo mật Cao: Mật mã yếu (Hạng: High)}
\textbf{Vấn đề:} Class \texttt{PasswordEncryptionService.java} chịu trách nhiệm mã hóa mật khẩu nhưng sử dụng các thuật toán và thông số đã lỗi thời, không còn an toàn:
\begin{itemize}
    \item \textbf{Thuật toán yếu:} Sử dụng \texttt{PBKDF2WithHmacSHA1}. Thuật toán SHA-1 đã bị xem là không an toàn, nên sử dụng SHA-256 hoặc SHA-512.
    \item \textbf{Số vòng lặp thấp:} Chỉ sử dụng 20000 vòng lặp. OWASP khuyến nghị số vòng lặp tối thiểu là 100,000 hoặc cao hơn.
    \item \textbf{Salt size nhỏ:} Kích thước salt chỉ là 8 bytes (64 bit), nên tăng lên ít nhất 16 bytes (128 bit) để chống lại các cuộc tấn công rainbow table.
\end{itemize}

\textbf{Đề xuất:} Nâng cấp ngay lập tức dịch vụ mã hóa mật khẩu:
\begin{itemize}
    \item Thay thế \texttt{PBKDF2WithHmacSHA1} bằng \texttt{PBKDF2WithHmacSHA512}.
    \item Tăng số vòng lặp (iteration count) lên ít nhất 100,000.
    \item Tăng kích thước salt (salt size) lên 16 hoặc 32 bytes.
\end{itemize}

\section{Xử lý ngoại lệ không đầy đủ (Hạng: Medium)}
\textbf{Vấn đề:} Lỗi này xuất hiện rất phổ biến (ví dụ: \texttt{Register.java}, \texttt{Statistics.java}). Khi một ngoại lệ (Exception) xảy ra, mã nguồn chỉ gọi \texttt{e.printStackTrace()}. Điều này không cung cấp thông báo lỗi thân thiện cho người dùng và không ghi lại lỗi một cách có hệ thống, khiến việc debug và bảo trì rất khó khăn.

\textbf{Đề xuất:}
\begin{itemize}
    \item Thay thế \texttt{e.printStackTrace()} bằng một framework ghi log (ví dụ: Log4j, SLF4J).
    \item Đối với các lỗi trong servlet, hãy chuyển hướng (forward) người dùng đến một trang lỗi (ví dụ: \texttt{error.jsp}) với thông báo lỗi rõ ràng.
\end{itemize}

\section{Nguy cơ NullPointerException (Hạng: Medium)}
\textbf{Vấn đề:} Nhiều phần của mã nguồn không kiểm tra giá trị \texttt{null} trước khi sử dụng, đặc biệt là khi lấy dữ liệu từ \texttt{request.getParameter()} (ví dụ: \texttt{Login.java}, \texttt{BookUpdate.java}) hoặc khi các phương thức DAO trả về \texttt{null}.

\textbf{Đề xuất:} Luôn luôn kiểm tra giá trị \texttt{null} đối với các biến có thể là \texttt{null} trước khi gọi phương thức trên chúng. Sử dụng \texttt{if (variable != null)} hoặc các kỹ thuật so sánh chuỗi an toàn như \texttt{"constant".equals(variable)}.

\section{Vi phạm quy ước và thiết kế (Hạng: Low)}
\textbf{Vấn đề:}
\begin{itemize}
    \item \textbf{Naming Convention:} Nhiều thuộc tính và phương thức sử dụng \texttt{snake\_case} (ví dụ: \texttt{book\_name}, \texttt{sum\_paid}) thay vì \texttt{camelCase} (ví dụ: \texttt{bookName}, \texttt{sumPaid}) theo chuẩn của Java.
    \item \textbf{Thiếu triển khai:} Toàn bộ class \texttt{Manager.java} chỉ chứa các phương thức rỗng, không có logic.
    \item \textbf{Thiếu Javadoc:} Hầu hết các class và phương thức đều thiếu Javadoc comment, gây khó khăn cho việc hiểu mục đích của mã.
\end{itemize}

\textbf{Đề xuất:} Thực hiện một đợt refactor toàn diện để đổi tên các biến/phương thức theo chuẩn \texttt{camelCase} và bổ sung Javadoc cho các class quan trọng.

\newpage